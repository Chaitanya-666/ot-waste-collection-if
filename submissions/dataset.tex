\documentclass[a4paper, 12pt]{article}
\usepackage[utf8]{inputenc}
\usepackage{geometry}
\usepackage{graphicx}
\usepackage{hyperref}
\usepackage{listings}
\usepackage{xcolor}

\geometry{a4paper, margin=1in}
\hypersetup{colorlinks=true, urlcolor=blue}

\definecolor{codegreen}{rgb}{0,0.6,0}
\definecolor{codegray}{rgb}{0.5,0.5,0.5}
\definecolor{codepurple}{rgb}{0.58,0,0.82}
\definecolor{backcolour}{rgb}{0.95,0.95,0.92}

\lstdefinestyle{mystyle}{
    backgroundcolor=\color{backcolour},   
    commentstyle=\color{codegreen},
    keywordstyle=\color{magenta},
    numberstyle=\tiny\color{codegray},
    stringstyle=\color{codepurple},
    basicstyle=\ttfamily\footnotesize,
    breakatwhitespace=false,         
    breaklines=true,                 
    captionpos=b,                    
    keepspaces=true,                 
    numbers=left,                    
    numbersep=5pt,                  
    showspaces=false,                
    showstringspaces=false,
    showtabs=false,                  
    tabsize=2
}
\lstset{style=mystyle}

\title{Dataset Generation for VRP-IF}
\author{Chaitanya Shinde (231070066) \and Harsh Sharma (231070064)}
\date{\today}

\begin{document}

\maketitle

\begin{abstract}
This document describes the data generation process for the Vehicle Routing Problem with Intermediate Facilities (VRP-IF). As no standard external dataset was used, a synthetic data generator was created to produce a wide range of solvable and realistic problem instances for testing and benchmarking the ALNS solver.
\end{abstract}

\section{Data Generation Mechanism}
The data generation is handled by the \texttt{DataGenerator} class located in \texttt{src/data_generator.py}. This class is responsible for creating a \texttt{ProblemInstance} object populated with a depot, a specified number of customers, and a specified number of intermediate facilities, all with randomly generated attributes.

The generation process is seeded to ensure that test instances are reproducible, which is crucial for consistent testing and debugging.

\subsection{Core Functionality}
The main entry point is the static method \texttt{generate_instance(...)}, which takes several parameters to control the characteristics of the generated problem.

\begin{lstlisting}[language=Python]
@staticmethod
def generate_instance(
    name: str,
    n_customers: int,
    n_ifs: int,
    area_size: int = 100,
    vehicle_capacity: int = 50,
    demand_range: Tuple[int, int] = (5, 15),
    cluster_factor: float = 0.0,
    seed: int = 42
) -> ProblemInstance:
    # ... implementation ...
\end{lstlisting}

\subsection{Parameters}
\begin{itemize}
    \item \texttt{name}: A string identifier for the instance.
    \item \texttt{n_customers}: The number of customers to generate.
    \item \texttt{n_ifs}: The number of intermediate facilities to generate.
    \item \texttt{area_size}: The size of the square area (e.g., 100x100) in which locations are placed.
    \item \texttt{vehicle_capacity}: The capacity of each vehicle.
    \item \texttt{demand_range}: A tuple specifying the minimum and maximum demand for each customer.
    \item \texttt{cluster_factor}: A float between 0.0 and 1.0 that controls the geographic distribution of customers. A factor of 0.0 results in a uniform distribution, while a factor closer to 1.0 results in more tightly clustered customers.
    \item \texttt{seed}: An integer used to seed the random number generator for reproducibility.
\end{itemize}

\section{Location Generation}
\begin{itemize}
    \item \textbf{Depot:} A single depot is created, typically at or near the center of the defined area.
    \item \textbf{Customers:} Customer locations are generated randomly within the area. If a \texttt{cluster_factor} is provided, a number of cluster centers are first generated, and customers are then placed in a normal distribution around these centers. This allows for the creation of more realistic, clustered problem instances. Each customer is assigned a random demand within the specified \texttt{demand_range}.
    \item \textbf{Intermediate Facilities (IFs):} IF locations are also generated randomly within the area, typically distributed to provide good coverage.
\end{itemize}

\section{Usage}
The data generator is used extensively in the test suites (\texttt{comprehensive_test_suite.py} and \texttt{tests/test_all.py}) to create a variety of problem instances for validating the solver's correctness and performance across different scales and complexities.

\begin{lstlisting}[language=Python]
from src.data_generator import DataGenerator

# Generate a medium-sized problem instance
problem = DataGenerator.generate_instance(
    name="Medium-Test",
    n_customers=15,
    n_ifs=2,
    seed=123
)

print(f"Generated problem with {len(problem.customers)} customers.")
\end{lstlisting}

This on-the-fly data generation capability was a core design choice, as it allows for flexible and comprehensive testing without reliance on external data files.

\end{document}
