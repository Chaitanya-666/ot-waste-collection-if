\documentclass{article}
\usepackage[utf8]{inputenc}
\usepackage{amsmath}
\usepackage{amsfonts}
\usepackage{amssymb}
\usepackage{graphicx}
\usepackage{hyperref}
\usepackage{booktabs} % For better table formatting

\title{Result Sheet}
\author{}
\date{}

\begin{document}

\maketitle

\section{Result Sheet}

This document outlines the expected structure and content of the results generated by running the project's comprehensive test suite. The \texttt{comprehensive\_test\_suite.py} script is designed to execute a series of predefined test cases, each evaluating the Adaptive Large Neighborhood Search (ALNS) algorithm under different problem complexities and hyperparameter configurations. For each test case, the script generates an optimization video and logs various performance metrics.

The actual results are dynamically generated upon execution of \texttt{comprehensive\_test\_suite.py}. The table below provides an example of the information that would be populated in this result sheet after running the tests.

\section{Test Case Summary}

The comprehensive test suite includes a variety of scenarios to assess the ALNS algorithm's performance across different scales and problem characteristics. Each test case typically involves:

\begin{itemize}
    \item \textbf{Problem Generation:} Using \texttt{src/data\_generator.py} with specific parameters (e.g., number of customers, cluster factor).
    \item \textbf{ALNS Solver Configuration:} Setting hyperparameters such as maximum iterations, initial temperature, and cooling rate.
    \item \textbf{Solution Evaluation:} Calculating the total cost (e.g., total distance) of the optimized routes.
    \item \textbf{Visual Output:} Generating a GIF animation showcasing the evolution of routes during the optimization process.
\end{itemize}

\section{Example Results Table}

The following table illustrates the kind of output that \texttt{comprehensive\_test\_suite.py} produces. The 'Obtained Result', 'Optimal?', 'Status', 'Time', and 'Outputs' columns would be populated with the actual values after the test suite is executed.

\begin{table}[htbp]
    \centering
    \caption{Example Test Results}
    \begin{tabular}{p{2.5cm} p{3.5cm} p{3.5cm} p{1.5cm} p{1.5cm} p{1cm} p{1.2cm} p{1cm} p{2.5cm}}
        \toprule
        \textbf{Test Case ID} & \textbf{Parameters} & \textbf{Hyperparameters} & \textbf{Expected Result} & \textbf{Obtained Result} & \textbf{Optimal?} & \textbf{Status} & \textbf{Time} & \textbf{Outputs} \\
        \midrule
        \texttt{test\_01\_tiny\_fast} & \texttt{name: Tiny-Fast}\newline\texttt{n\_customers: 5}\newline\texttt{n\_ifs: 1}\newline\texttt{seed: 42} & \texttt{max\_iterations: 20}\newline\texttt{temperature\_initial: 500}\newline\texttt{cooling\_rate: 0.99} & \texttt{<= 300} & (e.g., \texttt{285.50}) & (e.g., \texttt{Yes}) & (e.g., \texttt{\checkmark Pass}) & (e.g., \texttt{0.50s}) & \href{submissions/test_01_tiny_fast.gif}{GIF} \\
        \texttt{test\_02\_tiny\_deep} & \texttt{name: Tiny-Deep}\newline\texttt{n\_customers: 5}\newline\texttt{n\_ifs: 1}\newline\texttt{seed: 42} & \texttt{max\_iterations: 100}\newline\texttt{temperature\_initial: 1000}\newline\texttt{cooling\_rate: 0.98} & \texttt{<= 250} & (e.g., \texttt{240.25}) & (e.g., \texttt{Yes}) & (e.g., \texttt{\checkmark Pass}) & (e.g., \texttt{1.20s}) & \href{submissions/test_02_tiny_deep.gif}{GIF} \\
        \texttt{test\_03\_small\_fast} & \texttt{name: Small-Fast}\newline\texttt{n\_customers: 10}\newline\texttt{n\_ifs: 2}\newline\texttt{seed: 42} & \texttt{max\_iterations: 50}\newline\texttt{temperature\_initial: 1000}\newline\texttt{cooling\_rate: 0.99} & \texttt{<= 600} & (e.g., \texttt{580.75}) & (e.g., \texttt{Yes}) & (e.g., \texttt{\checkmark Pass}) & (e.g., \texttt{1.80s}) & \href{submissions/test_03_small_fast.gif}{GIF} \\
        \texttt{test\_04\_small\_deep} & \texttt{name: Small-Deep}\newline\texttt{n\_customers: 10}\newline\texttt{n\_ifs: 2}\newline\texttt{seed: 42} & \texttt{max\_iterations: 200}\newline\texttt{temperature\_initial: 5000}\newline\texttt{cooling\_rate: 0.985} & \texttt{<= 500} & (e.g., \texttt{490.10}) & (e.g., \texttt{Yes}) & (e.g., \texttt{\checkmark Pass}) & (e.g., \texttt{4.50s}) & \href{submissions/test_04_small_deep.gif}{GIF} \\
        \texttt{test\_05\_medium\_fast} & \texttt{name: Medium-Fast}\newline\texttt{n\_customers: 15}\newline\texttt{n\_ifs: 2}\newline\texttt{seed: 42} & \texttt{max\_iterations: 100}\newline\texttt{temperature\_initial: 2000}\newline\texttt{cooling\_rate: 0.99} & \texttt{<= 800} & (e.g., \texttt{790.30}) & (e.g., \texttt{Yes}) & (e.g., \texttt{\checkmark Pass}) & (e.g., \texttt{3.00s}) & \href{submissions/test_05_medium_fast.gif}{GIF} \\
        \texttt{test\_06\_medium\_deep} & \texttt{name: Medium-Deep}\newline\texttt{n\_customers: 15}\newline\texttt{n\_ifs: 2}\newline\texttt{seed: 42} & \texttt{max\_iterations: 300}\newline\texttt{temperature\_initial: 10000}\newline\texttt{cooling\_rate: 0.985} & \texttt{<= 700} & (e.g., \texttt{680.90}) & (e.g., \texttt{Yes}) & (e.g., \texttt{\checkmark Pass}) & (e.g., \texttt{9.00s}) & \href{submissions/test_06_medium_deep.gif}{GIF} \\
        \texttt{test\_07\_large\_fast} & \texttt{name: Large-Fast}\newline\texttt{n\_customers: 25}\newline\texttt{n\_ifs: 3}\newline\texttt{seed: 42} & \texttt{max\_iterations: 150}\newline\texttt{temperature\_initial: 5000}\newline\texttt{cooling\_rate: 0.99} & \texttt{<= 1200} & (e.g., \texttt{1180.40}) & (e.g., \texttt{Yes}) & (e.g., \texttt{\checkmark Pass}) & (e.g., \texttt{7.00s}) & \href{submissions/test_07_large_fast.gif}{GIF} \\
        \texttt{test\_08\_large\_deep} & \texttt{name: Large-Deep}\newline\texttt{n\_customers: 25}\newline\texttt{n\_ifs: 3}\newline\texttt{seed: 42} & \texttt{max\_iterations: 500}\newline\texttt{temperature\_initial: 20000}\newline\texttt{cooling\_rate: 0.985} & \texttt{<= 1000} & (e.g., \texttt{980.50}) & (e.g., \texttt{Yes}) & (e.g., \texttt{\checkmark Pass}) & (e.g., \texttt{20.00s}) & \href{submissions/test_08_large_deep.gif}{GIF} \\
        \texttt{test\_09\_clustered\_data} & \texttt{name: Clustered}\newline\texttt{n\_customers: 20}\newline\texttt{n\_ifs: 2}\newline\texttt{seed: 42}\newline\texttt{cluster\_factor: 0.8} & \texttt{max\_iterations: 300}\newline\texttt{temperature\_initial: 10000}\newline\texttt{cooling\_rate: 0.99} & \texttt{<= 900} & (e.g., \texttt{870.60}) & (e.g., \texttt{Yes}) & (e.g., \texttt{\checkmark Pass}) & (e.g., \texttt{8.00s}) & \href{submissions/test_09_clustered_data.gif}{GIF} \\
        \texttt{test\_10\_uniform\_data} & \texttt{name: Uniform}\newline\texttt{n\_customers: 20}\newline\texttt{n\_ifs: 2}\newline\texttt{seed: 42}\newline\texttt{cluster\_factor: 0.0} & \texttt{max\_iterations: 300}\newline\texttt{temperature\_initial: 10000}\newline\texttt{cooling\_rate: 0.99} & \texttt{<= 1100} & (e.g., \texttt{1080.70}) & (e.g., \texttt{Yes}) & (e.g., \texttt{\checkmark Pass}) & (e.g., \texttt{8.50s}) & \href{submissions/test_10_uniform_data.gif}{GIF} \\
        \bottomrule
    \end{tabular}
\end{table}

\section{Interpretation of Results}

The key metrics to observe in the results are:

\begin{itemize}
    \item \textbf{Obtained Result:} The final total cost (e.g., total distance) of the best solution found by the ALNS algorithm for a given test case.
    \item \textbf{Optimal?:} Indicates whether the obtained result is within the \texttt{expected\_cost\_upper\_bound}. For complex VRPs, finding a globally optimal solution is often intractable, so "optimal" in this context usually means "within an acceptable range of a known best solution or upper bound".
    \item \textbf{Time:} The execution time of the ALNS algorithm, providing an indication of its computational efficiency for different problem sizes and ALNS configurations.
    \item \textbf{Outputs:} Links to visual aids (GIFs) that illustrate the optimization process, showing how the vehicle routes evolve and improve over iterations. These visualizations are crucial for qualitative analysis of the algorithm's behavior.
\end{itemize}

By analyzing these results across various test cases, one can assess the effectiveness, efficiency, and robustness of the implemented ALNS algorithm for waste collection route optimization.

\end{document}
